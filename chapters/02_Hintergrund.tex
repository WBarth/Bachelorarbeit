\section{Hintergrund}
\label{Hintergrund}
Der Fußschalter soll in einer komplexen Umgebung aus Messwerkzeug und Computer agieren, weshalb in diesem Kapitel zum Verständnis wichtige Hintergrundinformationen gegeben werden. Zunächst wird \ac{BLE} vorgestellt, dass die Grundtechnologie für den Informationsaustausch zwischen Fußschalter und dem Messwerkzeug darstellt sowie die \ac{HCT}-Plattform und das \ac{HCT}-Protokoll auf dem sie aufbaut.

\subsection{Bluetooth Low Energy}
\ac{BLE} ist eine Funktechnologie, die in Hinsicht auf einen sehr geringen Energieverbrauch entwickelt wurde. Sie arbeitet im 2.4 GHz unlicensed \ac{ISM} frequency band. (\cite[]{Bluetooth_Wireless_Technology}) \ac{BLE} findet hauptsächlich Verwendung darin, eine einfache ``kabellose Verbindung zwischen einem Smartphone und Audioendgeräten'' herzustellen sowie die ``Anbindung von kabellosen Tastaturen und anderen Eingabegeräten an Notebooks, PCs und Smartphones'' zu ermöglichen. (\cite[S. 339]{Grundkurs_mobile_Kommunikation})\\
Die Übertragung von gerätespezifischen Daten, wie die gemessene Temperatur bei einem digitalen Thermometer, erfolgt dabei über die Mechanismen des \ac{GATT}, welche Services, Charakteristiken und deren Values definiert. Diese werden während des Verbindungsaufbaus im Zuge der Service Discovery übermittelt und ein Server kann sich auf die Charakteristik anmelden, was als Subscription bezeichnet, sodass er über eine Änderung der Attribute vom Client selbstständig notifiziert wird. Andernfalls kann direkt über einen Read Request die Werte einer Charakteristik angefordert werden. Mehrere Charakteristiken werden in einem Service zusammengefasst (\cite[S. 1459]{Bluetooth_Core_Specification})(\cite[]{Overview_and_Evaluation_of_BLE}). Der Protokoll Stack von Bluetooth hält sich ``lose an das 7 Schichten OSI Modell'' (\cite[S. 347]{Grundkurs_mobile_Kommunikation}).\\
Innerhalb einer Verbindung über Bluetooth werden den Geräten verschiedene Rollen zugeordnet. Das Gerät, das den Verbindungsaufbau initiiert hat, wird als das ``Central'' bezeichnet und verarbeitet die Daten des ``Peripheral''. Das Peripheral ist dabei meist ein Gerät, welches streng limitiert durch seine ihm zur Verfügung stehenden Ressourcen ist und stellt seine Daten dem Central zur Verfügung. Wird zum Beispiel ein digitales Thermometer mit einem Handy verbunden, um die Temperatur abzulesen, ist somit das Thermometer das Peripheral und das Handy das Central. Neben den Rollen Central und Peripheral, sind weiterhin die Rollen ``Broadcaster'' und ``Observer'' vorhanden, diese haben jedoch für diese Arbeit keine Bedeutung (\cite[S. 1246]{Bluetooth_Core_Specification}).\\
Bei den Werkzeugen der Hoffmann Group wurde sich für \ac{BLE} als Verbindungsmedium entschieden, da den Hand- und Messwerkzeugen, aufgrund ihres Formfaktors, nur begrenzte Batteriespeicherkapazitäten zu Verfügung stehen.

\subsection{HCT-Plattform}
Um die Digitalisierung der Messergebnisse dem Anwender so einfach wie möglich zu gestalten, setzt die Hoffmann Group mit der \ac{HCT}-Plattform darauf, die Digitalisierung der durchgeführten Arbeitsschritte als einen festen Bestandteil in ihr Werkzeug zu integrieren. Diese sind zum Stand dieser Arbeit folgende Werkzeuge: 
\begin{itemize}
	\item Drehmomentschlüssel
	\item Messschieber
	\item Messuhren
	\item Drehmomentprüfgerät
	\item Bügelmessschrauben
\end{itemize}
Die Werkzeuge stellen dem Anwender die Daten der Messungen in verschiedener Weise zu Verfügung. Zum einen können die Geräte als ein \ac{HID} über \ac{BLE} mit dem Computer verbunden werden. Sie simulieren dann eine über Bluetooth verbundene Tastatur, über die, die Messergebnisse als Tastendrücke serialisiert werden. Das Messergebnis kann dann in einem Texteditor oder Tabellenprogramm aufgefangen werden. Des Weiteren erzeugen die Drehmomentschlüssel und das Drehmomentprüfgerät eine \ac{CSV}-Datei, in der alle durchgeführten Messungen mit einer großen Anzahl an zusätzlichen Daten, wie dem Datum der Durchführung oder der Seriennummer des Werkzeugs gespeichert werden. Wird das Gerät über \ac{USB} mit dem Computer verbunden, zeigt es sich als \ac{MSC}-Device und die Datei kann per Drag-and-drop auf den Computer kopiert werden. Eine weitere Möglichkeit die durchgeführten Messungen zu digitalisieren, ist mithilfe der \ac{HCT}-Windows-App. Diese erfordert zusätzlich zur frei verfügbaren Software einen speziellen Hardwareschlüssel, genannt Dongle, der zum Verbinden der Geräte benötigt wird. Sie werden ebenfalls über \ac{BLE} verbunden und sprechen über \ac{BLE} das firmeneigene \ac{HCT}-Protokoll. Die Windows-App bietet zahlreiche Möglichkeiten die Messdaten zu digitalisieren und den Produktionsprozess zu optimieren. In der App können Schraubfälle angelegt werden und mit Bildern hinterlegt werden. Die Seriennummer von Werkstücken kann automatisch mit dem dazugehörigen Messwert verlinkt werden und \ac{CAQ}-Software kann über einen virtuelle COM-Port angebunden werden. Die \ac{HCT}-Windows-App unterstützt derzeit lediglich die Drehmomentschlüssel, jedoch ist die Einbindung der restlichen \ac{HCT}-Geräte bereits in der Entwicklung. Der Fußschalter und der Dongle nehmen dabei eine ähnliche Position, allerdings mit geringerem Funktions- und Userinterfaceumfang, wie die Windows-App, ein. Der letzte Baustein der \ac{HCT}-Plattform ist die \ac{HCT}-Mobile-App, welche ebenfalls frei erhältlich ist und vor allem die Bedienung des Geräts erleichtert, zum Beispiel bei Arbeitsschritten bei denen das Display des Werkzeugs für den Anwender nicht sichtbar ist.

\begin{figure}[H] 
	\centering
	\includegraphics[width=\textwidth]{figures/HCT-Plattform.png}
	\caption{Schematische Zeichnung der \ac{HCT}-Plattform}
\end{figure}

\subsection{HCT-Protokoll}
Der \ac{HCT}-Plattform liegt das firmeneigene \ac{HCT}-Protokoll zugrunde. Dieses stellt sicher, dass alle Geräte der \ac{HCT}-Plattform stets kompatibel zu einander sind. Es ist ein binäres Protokoll, dass entwickelt wurde um über \ac{BLE} gesprochen zu werden und stellt ein virtuelles Speichermodel der Geräte da. Dabei besitzen Werkzeuge unterschiedlicher Produktreihen und Hersteller jeweils verschiedene Speichermodelle. Über sogenannte ``READ'' und ``WRITE'' Befehle auf die Speicheradressen kann dann der interne Zustand des Werkzeug abgefragt und verändert werden. Siehe Anhang \ref{appendix:HCT-Protocol-Description}. So können auch komplexe Operationen effizient über \ac{BLE} durchgeführt werden. Der virtuelle Speicher ist dabei in Datenblöcken unterteilt, welche die Daten in logische Gruppen zusammenfassen. Die Datenblöcke stellen die Referenzadressen zu spezifischen Dateneinträgen dar. In Abbildung \ref{fig:virtuellesSpeichermodell} ist ein Auszug aus dem Speichermodell der Messuhren bzw. Messschieber mit den erwähnten Datenblöcken und spezifischen Dateneinträgen zu sehen. Zudem stehen automatisierte Tools zur Verfügung mit deren Hilfe das Framework, welches die Kommunikation über das \ac{HCT}-Protokoll abstrahiert, um die Speichermodelle neu eingeführter Werkzeuge erweitert werden kann. Dieses Framework trägt den internen Namen ``nrf\_Base'' und stellt die Anwendungsschicht im Bluetooth-Stack dar.

\begin{figure}[H] 
	\centering
	\includegraphics[width=\textwidth]{figures/HCT_Protocol_DCDG.png}
	\caption{Auszug aus dem virtuellen Speichermodell für Messuhren und Messschieber}
	\label{fig:virtuellesSpeichermodell}
\end{figure}


\subsection{Stand der Technik}
Für die Anbindung von Messmitteln an einem Computer ist eine vielzahl von alternativen Lösungsansätzen vorhanden, welche jedoch schwerwiegende Nachteile besitzen und durch den Fußschalter aufgelöst werden. Zum besseren Verständnis werden in diesem Kapitel zwei alternativen zum Fußschalter aufgezeigt.\\
Ein Lösungsansatz ist der ``V-Mux'' der Firma Brecht. Dieser benötigt einerseits eine Software Installation und andererseits einen USB-Dongle, über welchen nicht nur \ac{HCT}-Werkzeuge angebunden werden können, sondern auch Geräte der Firmen Sylvac und Mitutoyo. Der V-Mux gibt die Messergebisse über einen virtuellen COM-Port in den Protokolle Euromux, MUX10, MUX50 und DMX16 aus, wodurch die Messergebisse durch eine \ac{CAQ}-Software verwertbar werden (\cite[5]{VMUX}). Der V-Mux unterstützt keine Serialisierung der Messergebnisse über \ac{HID}.\\
Die ``T-Box'' der Firma Bobe serialisiert die Messergebnisse von bis zu 12 über Kabel verbundenen Messmitteln in \ac{HID}-Ausgaben, wofür keine weitere Software benötigt wird und eine Kombination mit einem Fußtaster der Firma Bobe ist möglich (\cite{TBOX-Bobe}). Für die Umwandlung in ein durch eine \ac{CAQ}-Software verwertbares Format, wird einerseits anderes Gerät, die ``M-Box'', benötigt und andererseits ist die Installation einer Treibersoftware erforderlich. Auch an die M-Box müssen die Messmittel über Kabel angeschlossen werden (\cite{MBOX-Bobe}).