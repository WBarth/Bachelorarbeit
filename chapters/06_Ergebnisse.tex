\section{Ergebnisse}
Derzeit sind bis zu fünf Verbindungen mit Werkzeugen und eine Verbindung mit einem Computer möglich. Hier kommt der Chip der Anwendung bereits an seine Grenzen, was in Zeitverzögerungen bis zur Ausgabe des Messergebnis im Computer führt. Da die Anzeige der Gruppennummer auf den Messuhren lediglich bis fünf geht ist dies eine ausreichende Anzahl an Verbindungen, jedoch besteht das Potizial die Performance der Anwendung zu Verbessern und möglicherweise noch mehr Verbindungen zu ermöglichen.\\
Während der Modus \ac{USB}-\ac{HID} und \ac{CDC} zufriedenstellend funktionieren, kommt es im Modus \ac{BLE}-\ac{HID} dazu, dass teilweise eine Zeitverzögerung auftritt bis die ersten Tastendrücke des Messergebnis im Computer getätigt werden. Auch die Zeitspanne zwischen den einzelnen Tastendrücke ist nicht zufriedenstellend und variert. Dabei steht in Verdacht, dass die Connection Intervalle bei mehreren bestehenden Verbindungen nicht gut koordiniert werden, also die Status Pakete der einzelnen Verbindung gehäuft um einen Zeitpunkt auftreten. Besser wäre es die Pakete auf die gesamte Zeispanne des Connection Intervals gleichmäßig zu verteilen. Dadurch kann es dazu kommen, dass das Softdevice die Anwendung des Fußschalters für eine bemerkbare Zeit blockiert, da es eine höhere Priorität besitzt. Die Optimierung der Connection Intervalle ist jedoch Aufgabe des unterliegenden Frameworks nrf\_Base.
%TODO Bild
Die Anwendung wurde erweiterbar für neue Messmodi gehalten. Fest geplant ist dabei der bereits beschriebene Modus \ac{BLE}-Windows-App, jedoch sind noch weitere Modi denkbar, wie ein Modus in dem die durchgeführten Messungen in einer \ac{CSV}-Datei persistiert werden.\\
Die Koordinierung von mehreren Messuhren mit der Gruppenfunktion ist eine vollständig neue Entwicklung, die es in keinem anderen Produkt gibt. Das Feedback von ersten Testkunden fällt dazu positiv aus, jedoch werden sich Verbessungmöglichkeiten erst im weiteren Gebrauch herausstellen.\\
Die Detektierung von Änderungen an den Konfigurationsdateien birgt immernoch eine Reihe von Gefahren. Im Endprodukt wird es durch einen kleinen Schalter, der für den Lufttransport eingeführten wurde, möglich sein die Batterie des Fußschalters vom USB abzutrennen. Dadurch ist es möglich den Fußschalter vollständig von Außen auszuschalten. Werden die Daten durch den Computer auf das Massenspeichermedium geschrieben, wird dies dem Anwender über die LED angezeigt. Schaltet er den Fußschalter dennoch aus kommt es zum derzeitgen Implementierungssstand dazu, dass die Konfigurationsdateien korrumpiert werden. In diesem Fall ist es jedoch nicht sicher, in wie weit ein Verlust von Dateien verhindert werden kann. Eine Lösung des Problems könnte sein, beim erfolgreichen Einlesen der Dateien eine Sicherungskopie der Schlüsseldaten des Filesystems zu machen und diese zu Laden falls eine Inkosistenz festgestellt wird, was im Falle des Fußschalters aufgrund der geringen Größe des Massenspeichermediums durchaus praktikabel sein könnte. Jedoch sollte auch eine weiterführende Analyse durchgeführt werden, was der Stand der Technik in diesem Zusammenhang ist, da dieses Problem für alle Massspeicher Geräte existieren muss. Im Falle des Fußschalters können an einen Nutzer im industriellen Arbeitsumfeld zudem etwas höhere Ansprüche gestellt werden, als an einen privaten Anwender und es wird erwartet, dass der Anwender eine Nutzungsanleitung liest oder in den Gebrauch eingewiesen wird. Dennoch werden die Konfigurationsdateien zum Download auf der Webseite der Hoffmann Group verfügbar gemacht, sodass beschädigte Dateien ersetzt werden können.\\
Eine weitere zusätzliche Funktionalität die denkbar ist, jedoch als Niederprior eingestuft wurde, ist es die Konfigurationsdateien in \ac{HTML} neuzudesignen. Diese könnten dann in einem Browser angezeigt werden und würde es ermöglichen die Schlüssel der Attribute nicht editierbar zu machen, sowie drop-down-Menüs für Attribute mit begrenzten Auswahlmöglichkeiten einzuführen. Mithilfe von Javascriptcode könnte dann die Konfiguration in einer Datei persistiert und dann von der Anwendung einglesen werden. Diese Funktionalität wurde noch nicht weiter verfolgt, weil einerseits die zeitliche Beschränkung dieser Arbeit dies nicht zugelassen hat und weil sie anderseits nicht ohne eine Javascript Komponenten auskommt. Dies wurde von einem Kollegen in einem ersten Gespräch behauptet und wirft das Problem auf, ob die stark gesicherten Computerumgebungen in der Fertigung die Ausführung von Javascriptcode zulassen. Des weiteren wurde diese Verbesserung des grafischen Interface grundsätzlich als niederprior eingestuft, weil der Fußschalter im industriellen Bereich eingesetzt wird und das Ausmaß seiner Interaktion damit fraglich ist.