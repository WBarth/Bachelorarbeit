\section{Ergebnisse}
In diesem Kapitel werden die Ergebnisse der Implementierung vorgestellt. Zunächst werden die zentralen Funktionalitäten der Gruppenfunktion und HID vorgestellt. Daraufhin die Ergebnisse diskutiert und im Abschluss der Stand der Produktentwicklung vorgestellt.

\subsection{Evaluierung}


\subsubsection{Gruppenfunktion}
Bei der Gruppenfunktion war die größte Herausforderung der Implementierung, dass gerade in HID Modi die Reihenfolge der Gruppe eingehalten werden muss, damit die Messergebnisse in der Ausgabe wieder einem Werkzeug zuordbar sind. In Abbildung 7 ist die Ausgabe der Gruppenfunktion über den virtuelle COM-Port im Modus CDC zu sehen. Die Ausgabe wurde in Terminalprogramm RealTerm aufgefangen und die einzelnen Ausgaben der Gruppenfunktion durch rote Striche getrennt. Durch Betätigen des Fußtasters werden also jeweils die drei Zeilen zwischen den roten Strichen ausgegeben. Durch die Kanalnummern die jeweils am Anfang der Zeilen stehen sind die Messergebnisse den einzelnen Geräten zuordbar. Zusätzlich wurde die Messuhren auf Messwerte eingestellt, die zu ihren Kanalnummern und Gruppennummern korrespondieren. 
\begin{figure}[H] 
	\centering
	\includegraphics[width=\textwidth]{figures/GroupFeature.png}
	\caption{Durchführung von Messungen mit der Gruppenfunktion über CDC}
\end{figure}

Die gleichen Messungen sind in Abbildung 8 im Modus USB-HID durchgeführt. Durch Betätigung des Fußtasters wird jeweils eine Zeile Messwerte ausgegeben. Es ist ersichtlich, dass bei einer Vertauschung der Messergebnis, dieser Fehler, bei echten Messdaten die von Zeile zu Zeile jeweils unterschiedlich sind, nicht nachvollzogen werden kann. Da hier jedoch die Einhaltung der Reihenfolge gezeigt werden soll, sind die Messergebnisse immer die Gleichen und korrespondieren zu ihren Gruppennummern. Es wird deutlich, dass auch bei mehreren durchgeführten Gruppenmessungen die Reihenfolge der Gruppe eingehalten wird.
\begin{figure}[H] 
	\centering
	\includegraphics[width=\textwidth]{figures/USBHIDGroup.png}
	\caption{Durchführung von Messungen mit der Gruppenfunktion über HID}
\end{figure}

\subsubsection{Zeitmessungen HID}
Für die Modi USB-HID und BLE-HID ist die Zeit die verstreicht bis das gesamte Messergebnis eingetippt wurde das wichtigeste Gütekriterium der Implementierung neben der offentsichtlichen Fehlerfreiheit. In Abbildung 9 sind daher die Logausgaben der Aufrufe zur Funktion die, die einzelnen Tastendrücke an die USB-HID-Library übergibt zu sehen. In Klammern ist die verstrichene Zeit seit dem ersten Tastendruck in Millisekunden zu sehen. Abgesehen von der Zeitspanne zwischen dem ersten Tastendruck und Tasten loslassen, in der weniger als eine Millisekunde vergangen ist, liegen zwischen jedem einzelnen Aufruf genau zwei Millisekunden. Es ist zu vermuten, dass die Zeitspanne zwischen den ersten beiden Aufrufen niedriger ist, da die Daten noch nicht über USB gesendet wurden, sondern zwischengespeichert sind. Sobald begonnen wird die Daten tatsächlich über USB zu senden, vergehen genau zwei Millisekunden. Das Messergebniss wird schnell und gleichmäßig über USB eingetippt. Zu den Zeitstempeln 22 und 24 wird dabei die Tabulatortaste gedrückt um in einer Tabelle in die nächste Spalte zu wechseln, während zum Zeitpunkt 34 und 36 die Zeile terminiert wird und in die nächste Zeile gesprungen wird.
\begin{figure}[H] 
	\centering
	\includegraphics[width=\textwidth]{figures/USBHID.png}
	\caption{Logging einer Ausgabe eines Messergebnis über USB-HID}
\end{figure}

Anders ist das zeitliche verhalten bei Eingabe des Messergebnis über BLE-HID. In Abbildung 10 ist der Mitschnitt der Nachrichten von einer verbundenen Messuhr, die beim Eintippen der Zeichen über BLE gesendet werden, zu sehen. Es wurde der Ellisys Bluetooth Analyzer verwendet. Hier ist auch zu sehen wie zum Zeitpunkt 13:30:35.883 das Messergebnis als HCT-Nachricht von der Messuhr an den Fußschalter übertragen wird. Ab dem Zeitpunkt 13:35.948 werden die Tastendrücke vom Fußschalter an den Computer übertragen. Dabei werden die Codes für Tastaturtasten übertragen und nicht die Codes der ASCII-Zeichen. Der Code 0 steht dabei für das loslassen einer Taste. Es wird deutlich, dass die einzelnen Tastendrücke mit unregelmäßigen Abständen eingetippt werden. Die Telegrame zwischen denen weniger als eine Millisekunde verstreicht werden dabei in einem Zeitslot übertragen. Es kommt in unregelmäßigen Zeitabständen dazu, dass mit ungefähr 110 Millisekunden eine verhältnismäßig lange Zeitspannen zwischen den einzelnen Nachrichten verstreicht, vorallem da das Connection Interval hier 30 Millisekunden beträgt und somit der Fußschalter deutlich mehr Zeitslots für weitere Übertragungen zur Verfügung hätte. Es wird beobachtet, dass die Ausgabe stottert.
\begin{figure}[H] 
	\centering
	\includegraphics[width=\textwidth]{figures/BLEHID1device.png}
	\caption{Mitschnitt der BLE Nachrichten eines BLE-HID Messergebnis von einer verbundenen Messuhr}
\end{figure}

Dieses Verhalten verschärft sich, wenn mehrere Messuhren verbunden sind. In Abbildung 11 ist der Mitschnitt der BLE-Nachrichten der Tastendrücke bei drei verbundenen Messuhren zu sehen. Die Übertragungen treten gehäuft auf. In einem Zeitslot werden mehrere Zeichen übertragen zwischen denen nur eine Millisekunde verstreicht, während die Zeit bis nur nächsten Übertragung mit 236 Millisekunden fast achtmal so groß wie das Connection Interval von 30 Millisekunden ist. Interessanterweise beträgt die Zeitdauer in allen der drei roten Kästen jeweils exakt 236 Millisekunden.
\begin{figure}[H] 
	\centering
	\includegraphics[width=\textwidth]{figures/BLEHID2device.png}
	\caption{Mitschnitt der BLE Nachrichten der BLE-HID Messergebnisse von drei verbundenen Messuhr}
\end{figure}

\subsection{Diskussion}
Die Koordinierung von mehreren Messuhren mit der Gruppenfunktion ist eine vollständig neue Entwicklung, die es in keinem anderen Produkt gibt und wird ein wichtiges Verkaufsargument für den Fußschalter sein. Bislang müssen Messuhren einzelnen entweder als HID-Device oder über die HCT-Windows-App verbunden werden. Auch das Feedback der ersten Testkunden fällt durchweg positiv zu dieser Funktionalität aus.\\
Während der Modus \ac{USB}-\ac{HID} und \ac{CDC} zufriedenstellend funktionieren, kommt es im Modus \ac{BLE}-\ac{HID} dazu, dass die Ausgabe stottert also in unregelmäßigen Abständen verhältnismäßig große Zeitspannen verstreichen bis die Ausgabe fortgeführt wird. Dabei steht in Verdacht, dass die Connection Intervalle bei mehreren bestehenden Verbindungen nicht gut koordiniert werden, also die Statuspakete der einzelnen Verbindung gehäuft um einen Zeitpunkt auftreten. Besser wäre es die Pakete auf die gesamte Zeispanne des Connection Intervals gleichmäßig zu verteilen. Dadurch kann es dazu kommen, dass das Softdevice die Anwendung des Fußschalters für eine bemerkbare Zeit blockiert, da es eine höhere Priorität besitzt. Die Optimierung der Connection Intervalle ist jedoch Aufgabe des unterliegenden Frameworks nrf\_Base.
%TODO Bild

\subsection{Stand Produktentwicklung}
Der Fußschalter und der Dongle sollen als eigenständige Produkte in das Sortiment der Hoffmann Group übernommen werden. Für sie werden jeweils Projekte vom Produktmanagement eröffnet. Dabei werden die Hardwareänderungen wie in Kapitel 5.4.1 beschrieben, in das Layout der Platine übernommen. Zusätzlich sollen die Pins, die den Chip in den Bootloader Mode versetzt, durch eine Taste von außen erreichbar gemacht werden. Außerdem sollen auch die SWE-Pads, die das Debugging auf dem Chip ermöglichen, erreichbar gemacht werden. Derzeitig sind sie auf der Seite mit der der Dongle auf die Platine gelötet wurde, weswegen sie nur sehr schwer abgreifbar sind. Die Prototypen dieser neuen Hardwareversion wurden gegen Ende dieser Arbeit erhalten und es wurde bereits festgestellt, dass die Taste die den Fußschalter in den Bootloadermodus versetzt nicht wie spezifiert funktioniert. Der Zulieferer der Platine muss diese daher erneut überarbeiten. Des weiteren werden umfassende Funk- und Reichweitenmessungen durchgeführt, da die ursprüngliche Platine dort nicht die Anforderungen erfüllte.\\
Der Modus \ac{BLE}-Windows-App wurde auf eine spätere Version verschoben, da die Integration der Messuhren und Messschieber in die Windows App andauert, sowie neue Geräte auf den Markt kommen sollen und deren Integration priorisiert wird. Eine Integration des Fußschalters ist daher noch nicht absehbar. Daher wurde die Implementierung des Modus zugunsten anderer Features bis auf Weiteres verschoben.\\
Insbesonders der Dongle ist bereits zu ersten Testläufen an ausgewählte Kunden gegeben worden und hat sich dort bereits bewährt. Desweiteren werden Codereviews mit anderen Entwicklern der Hoffmann Group durchgeführt.\\
Für die Funktionalität des Fußschalter beziehungsweise des Dongles wird eine Erfindungsmeldung herausgegeben und der Prozess der Patentierung der Technik wurde angestoßen. Siehe Anhang \ref{appendix:Erfindungsmeldung}.
