\section{Schlusswort}

\subsection{Stand Produktentwicklung}
Der Fußschalter und der Dongle sollen als eigenständige Produkte in das Sortiment übernommen werden. Sie werden jeweils als Projekte von dem Produktmanagement übernommen. Dabei werden die Hardwareänderungen wie in Kapitel 5.1 beschrieben, in das Layout der Platine übernommen. Zusätzlich soll der Pin, der den Chip in den Bootloader Mode versetzt, von außen erreichbar gemacht werden. Außerdem sollen auch die SWE-Pads, die das Debugging auf dem Chip ermöglichen, erreichbar gemacht werden. Derzeitig sind sie auf der Seite mit der der Dongle auf die Platine gelötet wurde, weswegen sie nur sehr schwer abgreifbar sind. Die Prototypen dieser neuen Hardwareversion wurden Anfang August erhalten.\\
Der Modus BLE-Windows-App wurde auf eine spätere Version verschoben, da die Integration der Messuhren in die Windows App andauert, sowie neue Geräte auf den Markt kommen sollen und deren Integration priorisiert wird. Eine Integration des Fußschalters ist daher noch nicht absehbar. Wegen dieses Grundes wurde dieser Modus zugunsten anderer Features bis auf Weiteres verschoben. 

\subsection{Fazit}
Der Implementierungsumfang für diese Arbeit war von Anfang an ambitioniert. Dennoch wurde alle geplanten Features außer dem Modus BLE-Windows-App implementiert und sogar noch zahlreiche zusätzliche Features, wie die Gruppenfunktion, umgesetzt. Eine weiterführende Optimierung des Modus BLE-HID steht noch aus, aber steht einer Produkteinführung nicht im Weg.