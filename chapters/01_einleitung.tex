\section{Einleitung}
In diesem Kapitel wird zunächst vorgestellt, wie genau die Problemstellung aussieht und welche Zielsetzung sich für den Fußschalter daraus ergibt.

\subsection{Problemstellung}
Die Digitalisierung von Hand- und Messwerkzeug in der Industrie ist in vollem Gange. Der Hauptgrund dafür ist, neben der einfacheren und genaueren Bedienung, die Möglichkeit durchgeführte Arbeitsschritte auf einem Computer automatisiert zu protokollieren. Wurden früher Messergebnisse per Hand vom Werkzeug auf Papier übertragen, werden sie nun zuverlässig und fehlerfrei über Bluetooth an einen Computer übertragen. Das steigert die Effizienz und ermöglicht die Automatisierung der Qualitätskontrolle, sowie den Nachweis, dass Standards in der Fertigung eingehalten wurden, was in Branchen wie der Automobilindustrie oder der Luftfahrt vom Gesetzgeber gefodert wird.\\
Diese Digitalisierung führt dazu, dass eine wachsende Anzahl von Geräten im Arbeitsumfeld der industriellen Fertigung ihre Messdaten zur Verarbeitung durch einen Computer zur Verfügung stellen. Dabei treten bei der Zusammenführung dieser Daten eine Reihe an praktischen Problemen auf. So muss zum Senden eines Messwerts an den Computer bei dem Messwerkzeug eine Taste gedrückt werden. Bei der Durchführung von hochpräzisen Messungen, die auf den hundertstel Millimeter genau sein müssen, verfälscht dieses Betätigen einer Taste auf dem Gerät jedoch bereits die Messung. Auch bei der Durchführung von möglichst zeitgleichen Messungen mit mehreren Messgeräten stellt das Drücken einer Taste zum Senden des Messwerts den Nutzer vor Probleme. Daher werden in der Industrie Fußschalter eingesetzt, die kabelgebunden sowohl an das Werkzeug als auch an den Computer, dieses Senden der Messung auslösen können. Kabelgebundene Fußschalter sind in ihrem Einsatzbereich beschränkt und in einem Produktionsumfeld unflexible einsetzbar.\\
Zudem wird die automatisierte Qualitätskontrolle mit \ac{CAQ}-Software durchgeführt, welche Arbeitsabläufe vorgeben und die Messergebnisse auf voreingestellte Toleranzen überprüfen kann. Sie benötigt die Messdaten des Werkzeugs in speziellen Protokollen über einen virtuellen COM-Port, weshalb Software- oder Hardware-Lösungen für die Umwandlung benötigt werden. Die softwarebasierten Lösungsansätze besitzen dabei den Nachteil, dass in der Fertigung strenge Sicherheitsrichtlinien gelten und deshalb für ihre Einführung zeitaufwändige und komplexe Prozesse durchgeführt werden müssen, welche die Einführung unattraktiv macht. Es wird daher dazu tendiert Hardware-Lösungen zu präferien, da sie keine Installation weiterer Software benötigen. Jedoch gibt es unter ihnen zum Stand dieser Arbeit noch keine Lösungen, die sowohl die Aufbereitung der Messdaten, als auch die kabellose Anbindung des Hand- und Messwerkzeug übernimmt.\\
Der Fußschalter stellt dabei einen Lösungsansatz dar, wie das Sammeln der Messdaten und Orchestrierung von mehreren in Gebrauch befindlichen Werkzeugen erleichtert und ermöglicht werden kann. Es wird die folgende Forschungsfrage gestellt: Wie sollten die Software- und Hardware-Komponenten eines drahtlosen Fußschalters gestaltet werden, damit dieser ohne die Installation weiterer Software Hand- und Messwerkzeug an ein Computersystem kabellose anbindet und deren Messergebisse in verschiedenen konfigurierbaren Formaten zur Verfügung stellt?

\subsection{Zielsetzung}
Der Fußschalter soll ohne einer Installation zusätzlicher Software die Messungen von Werkzeug erfassen und für die Weiterverarbeitung aufbereiten. Er soll dabei sowohl kabellos als auch über USB mit einem Computer verbindbar sein und der Anwender soll das Format in dem die Messdaten zur Verfügung gestellt wird, konfigurieren können. Durch die Entwicklung eines Fußschalters der sich über Bluetooth mit dem Messwerkzeug verbindet, soll der Einsatzbereich gegenüber kabelgebundenen Fußschaltern deutlich vergrößert und der Einsatz dem Anwender erleichtert werden. Durch die Taste des Fußschalters können die Messgeräten zeitgenau und ohnen den Messwert zu verfälschen angesteuert werden. Es soll dabei eine Funktionalität geschaffen werden, die es ermöglicht Messwerkzeuge zu einer Gruppe zusammenzufassen und ihre Messergebnisse durch einen einzigen Tastendruck abzufragen. Dabei muss insbesonders darauf geachtet werden, dass die Messdaten bei ihrer Ausgabe wieder den Werkzeugen zuzuordnen sind.\\
Aufgrund der Neuheit und Einzigartikeit des Fußschalters müssen neue Konzepte zur Orchestrierung und Datenverarbeitung der mehreren verbundenen Werkzeuge geschaffen werden. So muss der Verbindungsaufbau zu den Werkzeugen durchgeführt und weitere Informationen vom Werkzeug abgefragt werden. Damit verbunden steht die Herausforderung, dass der Fußschalter die unterschiedlich aufgebauten Daten der verschiedenen Werkzeuge verarbeiten und für zukünftig entwickelte Werkzeuge erweiterbar bleiben muss. Es wird sich auf das \ac{HCT}-Protokoll und \ac{HCT}-Werkzeug der Firma Hoffmann beschränkt, während Produkte anderer Firmen die nicht kompatibel mit dem \ac{HCT}-Protokoll sind von dem Lösungansatz nicht umfasst werden.\\
Für diese Ziele sollen im Laufe dieser Arbeit Lösungen erarbeitet, welche in Hinblick auf die verfügbare Hardware des Dongles und des Fußschalters umgesetzt werden. Dadurch soll für die Industrie ein Gerät geschaffen, welches die Datenverarbeitung von Messwerkzeugen signifikant erleichtert und bei einer Anbindung über \ac{HID} diese erst ermöglicht. Die erarbeiten Konzepte können für die Produkte und den damit verbundenen internen Protokollen von anderen Firmen abgewandelt oder erweitert werden, sowie Konzepte für generische Anbindung geschaffen werde.

\subsection{Struktur dieser Arbeit}
In Kapitel 2 werden Hintergrundinformationen gegeben, die für das Verständnis dieser Arbeit unerlässlich sind. Anschließend wird in Kapitel 3 der Ausgangszustand der Implementierung behandelt, da zu Beginn dieser Arbeit bereits ein großer Umfang an Code vorhanden ist, welcher abgegrenzt werden soll gegenüber den Erweiterungen, die für diese Arbeit durchgeführt wurden. Diese werden in Kapitel 4 methodisch und in Kapitel 5 praktisch vorgestellt. In Kapitel 6 werden dann die Ergebnisse der Implementierung diskutiert.