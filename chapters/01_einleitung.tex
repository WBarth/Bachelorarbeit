\section{Einleitung}
In diesem Kapitel wird zunächst vorgestellt, wie genau die Problemstellung aussieht und welche Zielsetzung sich für den Fußschalter daraus ergibt.

\subsection{Problemstellung}
Die Digitalisierung von Werkzeug in der Industrie ist in vollem Gange. Der Hauptgrund dafür ist, neben der einfacheren und genaueren Bedienung, die Möglichkeit durchgeführte Arbeitsschritte auf einem Computer automatisiert zu protokollieren. Wurden früher Messergebnisse per Hand vom Werkzeug auf Papier übertragen, werden sie nun zuverlässig und fehlerfrei über Bluetooth an einen Computer übertragen. Das steigert die Effizienz und ermöglicht die Automatisierung der Qualitätskontrolle, sowie den Nachweis, dass Standards in der Fertigung eingehalten wurden, was in Branchen wie der Automobilindustrie oder der Luftfahrt von großer Bedeutung ist.\\
Diese Digitalisierung führt dazu, dass eine wachsende Anzahl von Geräten im Arbeitsumfeld der Fertigung ihre Messdaten zur Verarbeitung durch einen Computer zur Verfügung stellen. Dabei treten bei der Zusammenführung dieser Daten eine Reihe an praktischen Problemen auf, die in den einzelnen Kapiteln genauer betrachtet werden. Der Fußschalter stellt dabei einen Lösungsansatz dar, wie das Sammeln der Messdaten und Orchestrierung von mehreren in Gebrauch befindlichen Werkzeugen erleichtert und ermöglicht werden kann.\\
Seit ihrer Einführung trifft die \ac{HCT}-Windows-App, die einen alternativen softwarebasierten Lösungsansatz für dieses Problem darstellt, immer wieder auf großen Widerstand von IT-Abteilungen, da die App direkt in der Fertigung eingesetzt werden muss, wo meist höchste Sicherheitsrichtlinen gelten. Zur Integrierung der Windows-App in den Fertigungsprozess müssen daher langwierige interne Prozesse angestoßen werden, die die Einführung unattraktiv machen.\\
Des Weiteren muss zum Senden eines Messwerts an den Computer bei dem Messwerkzeug eine Taste gedrückt werden. Bei der Durchführung von hochpräzisen Messungen, die auf den hundertstel Millimeter genau sein müssen, verfälscht dieses Betätigen einer Taste auf dem Gerät jedoch bereits die Messung. Auch bei der Durchführung von möglichst zeitgleichen Messungen mit mehreren Messgeräten stellt das Drücken einer Taste zum Senden des Messwerts den Nutzer vor Probleme. Daher werden in der Industrie Fußschalter eingesetzt, die kabelgebunden sowohl an das Werkzeug als auch an den Computer, dieses Senden der Messung auslösen können. Durch die kabelgebundene Natur dieser Fußschalter ist jedoch deren Einsatzbereich reduziert, da es nicht gegeben ist, dass in den Fertigungshallen und Werkstätten, in denen die Fußschalter eingesetzt werden, sich ein Computer in nächster Nähe befindet. 

\subsection{Zielsetzung}
Der Fußschalter soll ohne einer Installation zusätzlicher Software die Messungen von Werkzeug erfassen und in geeigneten Formaten dem Computer zur Weiterverarbeitung zur Verfügung stellen. Durch die Entwicklung eines Fußschalters der sich über Bluetooth mit dem Messwerkzeug verbindet, soll der Einsatzbereich gegenüber kabelgebundenen Fußschaltern deutlich vergrößert und der Einsatz dem Anwender erleichtert werden.\\
Aufgrund der Neuheit und Einzigartikeit des Fußschalters müssen neue Konzepte zur Orchestrierung und Datenverarbeitung der mehreren verbundenen Werkzeuge geschaffen werden.

\subsection{Struktur dieser Arbeit}
In Kapitel 2 werden Hintergrundinformationen gegeben, die für das Verständnis dieser Arbeit unerlässlich sind. Anschließend wird in Kapitel 3 der Ausgangszustand der Implementierung behandelt, da zu Beginn dieser Arbeit bereits ein großer Umfang an Code vorhanden ist, welcher abgegrenzt werden soll gegenüber den Erweiterungen, die für diese Arbeit durchgeführt wurden. Diese werden in Kapitel 4 vorgestellt und in Kapitel 5 die Ergebnisse diskutiert.