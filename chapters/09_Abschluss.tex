\section{Fazit}
Der Implementierungsumfang für diese Arbeit ist von Anfang an ambitioniert. Dennoch sind alle geplanten Features, außer dem Modus \ac{BLE}-Windows-App, implementiert. Zusätzlich sind zahlreiche zusätzliche Features, wie die Gruppenfunktion, entwickelt und implementiert. Während einige Funktionen, wie der Modus \ac{BLE}-\ac{HID}, noch Verbesserungspotenzial besitzen und für eine tatsächliche Produkteinführung ein umfangreiches Testen aller Funktionalitäten aussteht, ist die Erarbeitung der Konzepte zur Anbindung der Werkzeuge und Verarbeitung der Messdaten ein voller Erfolg, was sich einerseits durch die bereits erwähnte Erfindungsmeldung (Anhang \ref{appendix:Erfindungsmeldung}) und andereseits durch die Allokierung von Ressourcen durch das Produktmanagement zeigt.\\
Die Anwendung, die für diese Arbeit entwickelt wurde, ist auf die Werkzeuge der Hoffmann Group und damit verbunden auf das \ac{HCT}-Protokoll limitiert. Sie deckt zum Ende dieser Arbeit zwei wichtige Messmittel ab: den Drehmomentschlüssel und die Messuhr. Des Weiteren wurden Mechanismen geschaffen, um die verbleibenden und zukünftigen Messmittel der Hoffmann Group schnell und einfach einzubinden. Zudem sind Bestrebungen sowohl der Hoffmann Group als auch anderer Hersteller vorhanden, Werkzeuge, die nicht das \ac{HCT}-Protokoll spricht mit der \ac{HCT}-Plattform kompatibel zu machen. Während die Datenverarbeitung des Fußschalters und des Dongles weitestgehend unabhängig von dem \ac{HCT}-Protokoll ist, müssen für eine Erweiterungen auf andere Protokolle, zusätzliche Mechanismen und Konzepte geschaffen werden.\\
Während die Hoffmann Group den Fußschalter und den Dongle als Produkt verkaufen wird, kann die Forschung die Konzepte dieser Anwendung weiterführen und die Anbindung von nicht \ac{HCT}-fähigen Geräten ermöglichen. Beispielsweise ist die Hoffmann Group in diesem Zusammenhang an der Entwicklung der Bluetooth Special Interest Group (SIG) eines generischen Profils für industrielles Werkzeug beteiligt. Mit dem Fußschalter steht dabei ein Gerät bereit, das leicht darauf erweitert werden kann, Werkzeuge welche dieses Profil implementieren an einen Computer anzubinden.\\
Derzeit sind bis zu fünf Verbindungen mit Werkzeugen und eine Verbindung mit einem Computer möglich. Hier kommt der Chip der Anwendung bereits an seine Grenzen, was in Zeitverzögerungen bis zur Ausgabe des Messergebnisses im Computer führt. Da die Anzeige der Gruppennummer auf den Messuhren lediglich bis fünf geht, ist dies eine ausreichende Anzahl an Verbindungen. Allerdings besteht das Potenzial die Performance der Anwendung zu verbessern und möglicherweise noch mehr Verbindungen zu ermöglichen.\\
Die Anwendung ist erweiterbar für neue Messmodi gehalten. Fest geplant ist dabei der bereits beschriebene Modus \ac{BLE}-Windows-App. Ebenfalls sind noch weitere Modi denkbar, wie ein Modus in dem die durchgeführten Messungen in einer \ac{CSV}-Datei persistiert werden.\\
Die Detektierung von Änderungen an den Konfigurationsdateien birgt weiterhin eine Reihe von Gefahren. So wird es im Endprodukt durch einen kleinen Schalter, der für den Lufttransport eingeführten wird, die Möglichkeit bestehen die Batterie des Fußschalters vom USB abzutrennen. Dadurch ist der Nutzer in der Lage den Fußschalter vollständig von außen auszuschalten. Werden die Daten durch den Computer auf das Massenspeichermedium geschrieben, wird dies dem Anwender über die LED angezeigt. Schaltet dieser den Fußschalter dennoch aus kommt es im derzeitigen Implementierungssstand dazu, dass die Konfigurationsdateien korrumpiert werden. In diesem Fall ist nicht sicher, in wie weit ein Verlust von Dateien verhindert werden kann. Eine Lösung des Problems besteht darin, beim erfolgreichen Einlesen der Dateien eine Sicherungskopie der Schlüsseldaten des Filesystems zu erstellen und diese zu Laden, falls eine Inkonsistenz festgestellt wird, was im Falle des Fußschalters, aufgrund der geringen Größe des Massenspeichermediums durchaus praktikabel ist. Dennoch sollte auch eine weiterführende Analyse durchgeführt werden, was der Stand der Technik in diesem Zusammenhang ist, da dieses Problem für alle Massenspeicher Geräte existiert. Im Falle des Fußschalters können an einen Nutzer im industriellen Arbeitsumfeld höhere Ansprüche gestellt werden, als an einen privaten Anwender und es wird erwartet, dass der Anwender eine Nutzungsanleitung liest oder in den Gebrauch eingewiesen wird. Dennoch werden die Konfigurationsdateien zum Download auf der Webseite der Hoffmann Group verfügbar gemacht, sodass beschädigte Dateien austauschbar sind.\\
Eine weitere zusätzliche Funktionalität, die denkbar ist, jedoch als Niederprior eingestuft ist, ist es die Konfigurationsdateien in \ac{HTML} neuzudesignen. Diese können dann in einem Browser angezeigt werden und würde es ermöglichen die Schlüssel der Attribute nicht editierbar zu machen, sowie drop-down-Menüs für Attribute mit begrenzten Auswahlmöglichkeiten einzuführen. Mithilfe von Javascriptcode kann dann die Konfiguration in einer Datei persistiert und dann von der Anwendung eingelesen werden. Diese Funktionalität wird zum aktuellen Stand nicht weiter verfolgt, weil einerseits die zeitliche Beschränkung dieser Arbeit dies nicht zugelassen hat und andererseits die Funktionalität ersten Nachforschungen zufolge nicht ohne eine Javascript-Komponente zur Persistierung auskommt. Dabei besteht das Problem, dass wahrscheinlich die stark gesicherten Computerumgebungen in der Fertigung die Ausführung von Javascriptcode nicht zulassen. Des Weiteren ist diese Verbesserung des grafischen Interface grundsätzlich als Niederprior eingestuft, weil der Fußschalter im industriellen Bereich eingesetzt wird und das Ausmaß seiner Interaktion damit fraglich ist.\\
Die Hoffmann Group steht in engen Kontakt mit ihren Kunden und bindet deren Feedback direkt in die Produktentwicklung ein. Es ist davon auszugehen, dass aufgrund des User Feedbacks noch zahlreiche Fehlerbehebungen und Erweiterungen in den nächsten Jahren für den Fußschalter durchgeführt werden.