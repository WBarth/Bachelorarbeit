\section{Einbindung Messuhren}
Mit dem Fußschalter soll ein Gerät geschaffen werden, dass die Messdaten von einem möglichst breiten Spektrum an Werkzeug sammeln kann. Dabei bestehen grundlegende Abhängigkeiten zum Medium der Verbindung (\ac{BLE}) und zum Protokoll (\ac{HCT}), das über diese Verbindung gesprochen wird. Diese werden aufgrund der Spezifikation des Fußschalters nicht aufgelöst. Jedoch soll für diese Geräte Mechanismen geschaffen werden, um sie möglichst vollständig einzubinden und für in Zukunft entwickelte Geräte erweiterbar zu bleiben. Dabei gilt es außerdem die oft sehr unterschiedlichen Funktionen der Geräte im Fußschalter zu vereinen.\\
Daher wird auch in Vorbereitung auf die Implementierung der fußschalterspezifischen Features die Dongle-App um die Unterstützung der Messuhren bzw. Messschieber erweitert. Messuhren und Messschieber sind in diesem Kontext dabei gleich bedeutend, weil das \ac{HCT}-Interface für beide Geräte identisch ist. Es wird sich nachfolgend jedoch stets auf Messuhren bezogen, da sie für die Anwendungsfälle des Fußschalters und Dongles bedeutender sind.

\subsection{Unterscheidung der Geräte}
Bei der Einbindung von neuen Geräten in die Anwendung des Fußschalters zeigt sich schnell ein grundlegendes Problem. Während das \ac{HCT}-Protokoll den Verbindungsaufbau vollständig abstrahiert und damit alle \ac{HCT}-fähigen Geräte verbunden werden können, bestehen in den darüber hinausgehenden Daten schwerwiegende Unterschiede.\\
Im einfachsten Anwendungsfall ist der Fußschalter mit mehren Werkzeugen verbunden und sammelt passiven deren Messergebnisse auf. Unabhängig vom eingestellten Messmodus benötigt die Dongle-App von den Geräten einerseits das Messergebnis und andererseits die Messeinheit, um die Ergebnisse vollständig an den Computer weitergeben zu können. Diese befinden sich abhängig vom Gerät an folgenden \ac{HCT}-Protokoll-Adressen:

\begin{table}[H]
	\centering
	\begin{tabular}[H]{l|l|l}
		 & Datenblock & Adresse \\
		\hline
		Garant Drehmomentschlüssel & & \\
		Messergebnis & Measurement data block & 0x00000B04 \\
		Messeinheit & Setpoint data block & 0x00000C1F \\
		\hline
		Holex Drehmomentschlüssel & & \\
		Messergebnis & Device measurement result & 0x0000240C \\
		Messeinheit & Device measurement case config & 0x0000221B \\
		\hline
		Messuhren/Messschieber & & \\
		Messergebnis & Device measurement result & 0x00002408 \\
		Messeinheit & Device measurement case config & 0x0000220F \\
		
	\end{tabular}
	\caption{Adressen Messergebnis und Messeinheit}
\end{table}

In Tabelle 1 wird deutlich, dass die Adressen von Messergebnis und Messeinheit für diese drei Geräte jeweils unterschiedlich sind, auch wenn sie im Fall der Messuhren und Holex Drehmomentschlüssel im gleichen Datenblock liegen. Wird dieser als Binärdaten erhalten kann die Adresse des Datenblocks gelesen werden, die Information von welchem Gerät die Daten stammen und damit an welcher Stelle genau die jeweils die Daten zu finden sind, kann jedoch nicht festgestellt werden. Es bleibt also keine andere Möglichkeit als die verbundenen Geräte ihrem Typ zuzuordnen und die erhaltenen Daten entsprechend zu interpretieren.\\ 
Diese Unterscheidung mit welcher Art von Gerät kommuniziert wird, kann auf zwei verschiedenen Weisen erfolgen. Einerseits kann anhand des Namens das Gerät entweder der Klasse Drehmomentschlüssel oder Messuhr zugeordnet werden. Das ist jedoch unter Umständen fehleranfällig, da das Sortiment an \ac{HCT}-Werkzeug stetig wächst und es nicht auszuschließen ist, dass in der Zukunft Geräte auf den Markt kommen, mit deren Namen es zu falschen Zuordnung kommt. Zudem gibt es alte Messschieber, die zwar unter einem korrekten Name advertisen aber nicht das \ac{HCT}-Protokoll sprechen, wobei diese Geräte aufgrund der Inkompatibilität der Protokolle letztendlich nicht verbunden werden sollten.\\
Eine andere Methode ist, die ``Device Information'' abzufragen und anhand der Klassenidentifikationsnummer das Gerät einem Typ zuzuordnen. Das ist die präferierte Lösung, da neben der genaueren und sicheren Zuordnung, in der Device Information andere Informationen, wie die Protokoll Version mitgeliefert werden, die bei späteren Features unter Umständen benötigt werden. Zudem bleibt die Datenverarbeitung leichter erweiterbar um neue Geräte. Jedoch muss diese Abfrage in den asynchronen und mehrteiligen Ablauf des Verbindungsaufbaus eingefügt werden, wodurch ein höherer Entwicklungsaufwand entsteht. Auch benutzen Drehmomentschlüssel der Firma Garant eine andere Klassifikation der Geräte als das Werkzeug der Firma Holex, welche innerhalb der Dongle-App vereinheitlicht werden muss.\\
Da die Erweiterbarkeit und Flexibilität jedoch dem höheren Entwicklungsaufwands überwiegen, wurde sich zugunsten dieser Lösung entschieden. Dabei wird nach dem Verbindungszustand-Callback, der vom Central Device nach einer erfolgreichen Subscription der Anwendung auf die \ac{BLE}-Charakteristik des Werkzeugs aufgerufen wird, eine Nachricht zu Abfrage der Device Information gesandt. Die erhaltenen Daten werden dann in der zum Gerät gehörenden Struktur gespeichert und anschließend die Abfrage der Messeinheit durchgeführt. Dies muss mithilfe eines Debuggers überprüft werden, da vorerst die Ausgabe der Messergebnisse von Messuhren nicht funktioniert. Dazu werden einen Drehmomentschlüssel und eine Messuhr mit einem nRF52840 DK development board verbunden. Dann kann mit dem Debugger der IAR-Workbench \ac{IDE} direkt die Werte in der Gerätestrukturen überprüft werden.
%TODO: Bild 

\subsection{Anpassung der Messdatenverarbeitung}
Auch die Datenverarbeitung, also die Interpretation der erhalten Daten über \ac{BLE} muss angepasst werden, da das Messergebnis, wie bereits in Tabelle 1 gezeigt, zwar innerhalb des gleichen Datenblocks wie bei dem Holex Drehmomentschlüssel liegt, jedoch innerhalb des Datenblocks an einer anderen Stelle. Diese Offsets sind als Konstanten im Headerfile der Dongle-App definiert und müssen um die entsprechenden Einträge für die Messuhren erweitert werden. Werden die Daten erhalten wird als Erstes die Adresse ausgelesen und dann je nach Typ des zugehörigen Geräts, die Interpretierung der Daten durchgeführt. Sollen also weitere Geräte eingebunden werden, können zusätzliche Sonderbehandlungen der Daten anhand des Typs hinzugefügt werden. Zudem handelt es sich bei dem Messergebnis, das von Interesse ist, beim Drehmomentschlüssel um den ``Peak Torque'', der als Gleitkommazahl codiert ist, während bei der Messuhr das Ergebnis die ``Measurement Distance'' als Ganzzahl codiert ist. Dabei handelt es sich abhängig von der Messeinheit um Mikrometer oder Mikroinch. Es muss auf Millimeter beziehungsweise Inch umgerechnet werden, da die Ausgabe über CDC oder \ac{HID} der Anzeige auf dem Gerätedisplay gleichen soll. Das binäre Messergebnis wird daher zunächst mit einem Umrechnungsfaktor von 1000 in Millimeter beziehungsweise Milliinch als Gleitkommazahl umgerechnet. Anschließend muss überprüft werden, ob es sich bei der derzeitige Einheit des Geräts um Inch handelt, da dann der Wert erneut durch 1000 dividiert werden muss, um von Milliinch auf die gewünschte Einheit Inch zu gelangen. Es wird also letztendlich eine Gleitkommazahl erhalten, wodurch sie ohne Anpassungen in der Nachrichten Struktur der Dongle-App gespeichert werden kann.\\
Die Einheitenkodierung der Messuhr ist komplementär zur Kodierung der Drehmomentschlüssel und daher kann die if-Cascade, welche die Zuordnung vornimmt, um die Einheiten der Messuhr erweitert werden. Jedoch befindet sich die Information welche Einheit verwendet wird, wie das Messergebnis, ebenfalls an einer anderen Stelle innerhalb des Datenblocks und muss entsprechend angepasst werden.

\subsection{Gruppenfunktion}
Zwischen Messuhren und Drehmomentschlüssel besteht ein funktioneller Unterschied. Zwar messen beide Geräte kontinuierlich einen Wert, jedoch ist das eigentliche Messergebnis, das ausgegeben und gespeichert wird, ein jeweils anderes. Eine Messuhr oder Messschieber zeigt ausschließlich den derzeitigen Messwert an, während der in einer bestimmten Zeit maximal erreichte Messwert von keinem Interesse ist. Das ist bei einem Drehmomentschlüssel genau anders herum, da er nachdem eine Schraube angezogen wurde, die maximal erreichte Kraft, die während der Verschraubung erreicht wurde, als Messergebnis anzeigen soll und es auch solange anzeigt, bis erneut eine Kraft am Drehmomentschlüssel anliegt. Die Funktionalität des Fußtasters den Anwender zu befähigen ein Messergebnis zeitgenau und präzise auszulösen, hat bei der Art wie das Messergebnis bei Drehmomentschlüssel vorliegt, keine besondere Bedeutung. Es wurde sich daher entschieden bei Betätigung des Fußtasters nicht das Messergebnis von Drehmomentschlüssel abzufragen, sondern die Funktionalität ganz auf Messuhren zuzuschneiden.\\
Durch das Betätigen des Fußtasters des Fußschalters soll bei allen verbundenen Messuhren das derzeitige Messergebnis abgefragt werden. Während im Modus 2 (CDC) ein Messergebnisse, anhand der Kanalnummer einem Werkzeug zugeordnet werden kann, ist dies in den \ac{HID}-Modi nicht der Fall. Es muss daher die korrekte Reihenfolge der Ausgabe der Messergebnisse sichergestellt werden. Es wurde sich entschieden, das devices.csv Konfigurationsfile um eine Spalte mit einer Gruppennummer zu erweitern, da somit der Anwender sowohl die Reihenfolge als auch welche Messuhren in der Gruppe sind, konfigurieren kann. Da durch das Abschicken der Nachrichten zur Abfrage der Messungen in der korrekten Reihenfolgen, das tatsächliche Erhalten in der gleichen Reihenfolge nicht sichergestellt ist, muss davon ausgegangen werden, dass die Nachrichten in einer zufälligen Reihenfolgen erhalten werden. Stattdessen muss bei der Ausgabe die Nachrichten umsortiert werden. Dazu bedarf es eines Zählers, der durch Betätigung des Fußschalters, von 0 auf 1 gesetzt wird, wodurch der Start der Gruppenfunktion später beim Erhalten der Messergebnisse erkennbar ist. Bei der Abarbeitung der erhaltenen Nachrichten, wird dann über die Zuordnung zum Device, die Nachricht ausgewählt und weitergegeben, die zum Counter korrespondiert. Die Gruppenids, die keinem der konfigurierten Geräte zugeordnet werden können, sowie die Gruppenids die zu unverbundenen Geräten gehören, werden dabei übersprungen. Zusätzlich soll ein Feature der Messeruhren genutzt werden, um die Gruppennummer auch auf der Messuhr anzuzeigen. Dazu werden den Messuhren ihre Gruppennummer nach dem Verbindungsaufbau via dem \ac{HCT}-Protokoll übermittelt.\\
Durch spätere Anregungen von Messtechnikern ergab sich jedoch, dass wenn ein Gerät der Gruppe zwar konfiguriert, jedoch nicht verbunden ist, die präferierte Lösung ist die gesamte Gruppe nicht zu triggern. Das ergibt sich einerseits dadurch, dass die Messuhren sich aufgrund ihrer relativ kleine Batterien bei Inaktivität schnell ausschalten und die Verbindung zu ihrem Central trennen. Anderseits soll möglichst jede durchgeführte Messung korrekt sein, da sie durch die \ac{CAQ}-Software verarbeitet und gespeichert wird. Das Ziel der Sicherstellung einer korrekten Messung überwiegt hier also dem Gedanken der Benutzerfreundlichkeit, dass eine Messung auch dann gemacht werden kann, wenn einer der Geräte unverbunden ist. Der Fußschalter blinkt dann zwei Mal kurz rot auf, um diesen Fehler zu signalisieren.\\
Des Weiteren könnte es passieren, dass eine Messung zwar angefragt, aber nicht erhalten wurde. Die Anwendung würde dann blockieren, da auf die Nachricht gewartet wird und müsste neugestartet werden. Es wurde sich entschieden, einen Timer zu starten, falls eine Nachricht nicht erhalten wurde und bei seinem Ablaufen statt dem Messergebnis eine Fehlermeldung auszugeben. Die restlichen Messergebnisse können dann korrekt ausgegeben werden.\\
Ebenfalls aus dem Feedback der Messtechnikern heraus, wurde ein sequentielles Triggern der Gruppe implementiert. Dazu muss der entsprechende Eintrag in der Konfigurationsdatei mit einer Eins auf aktiv gesetzt werden. Es wird bei den Geräten der Gruppe nacheinander, jeweils bei Betätigen des Fußtaster, die Abfrage des Messergebnis ausgegelöst.