\section{Einbindung Messuhren}
In Vorbereitung auf die Implementierung der fußschalterspezifischen Features wird die Dongle-App um die Unterstützung der Messuhren bzw. Messschieber erweitert, da sich die meiste Funktionalität des Fußschalters spezielle auf die Messuhren bezieht. Messuhren und Messschieber können dabei gleich bedeutend verwendet werden, da das HCT-Interface für beide Geräte identisch ist. Bisher ist die Dongle App daraufhin ausgelegt, dass mit einem Drehmomentschlüssel kommuniziert wird und die erhaltenen Binärdaten werden entsprechen interpretiert. Für die Einbindung der Messuhren muss nun die Unterscheidung eingeführt werden mit welcher Art von Werkzeug kommuniziert wird und entsprechend die Kommunikation angepasst werden.

\subsection{Unterscheidung Geräte}
Die Unterscheidung mit welcher Art von Gerät kommuniziert wird, kann auf zwei verschiedenen Weisen erfolgen. \\
Einerseits kann anhand des Namens das Gerät entweder der Klasse Drehmomentschlüssel oder Messuhr bzw. Messschieber zugeordnet werden. Das ist jedoch unter Umständen fehleranfällig, da das Sortiment an HCT-Werkzeug stetig wächst und es nicht ausschließen ist, dass in der Zukunft Geräte auf den Markt kommen, mit deren Namen es zu falschen Zuordnung kommt. Zudem gibt es alte Messschieber, die zwar unter einem korrekten Name advertisen aber nicht das HCT-Protokoll sprechen, wobei diese Geräte aufgrund der Inkompatibilität der Protokolle letztendlich nicht verbunden werden sollten. \\
Eine andere Methode ist, die „Device Information“ abzufragen und anhand der Class ID das Gerät einem Typ zuzuordnen. Das ist die präferierte Lösung, da neben der genaueren und sicheren Zuordnung, in der Device Information andere Informationen, wie die Protokoll Version mitgeliefert werden, die bei späteren Features unter Umständen benötigt werden. Zudem bleibt die Datenverarbeitung leichter erweiterbar um neue Geräte.
Letztendlich wird nach dem Callback, der vom Central Device aufgerufen wird, wenn die erfolgreiche Subscription der Anwendung auf die BLE-Charakteristik des Werkzeugs stattgefunden hat, eine Nachricht zu Abfrage der Device Information gesandt. Die erhaltenen Daten werden dann in der zum Gerät gehörenden Struktur gespeichert.

\subsection{Korrigierung Abfrage Messeinheit}
Die Protokollbeschreibung der Messuhr zeigt, dass das Messergebnis zwar innerhalb des gleichen Datenblocks wie bei dem Holex Drehmomentschlüssel liegt, mit dem sich den gleichen Daten Callback teilt, jedoch innerhalb des Datenblocks an anderer Stelle. Diese Offsets sind als Konstanten im Headerfile der USB-App definiert und müssen um die entsprechenden Einträge für die Messuhren bzw. die Messschieber erweitert werden. Zudem handelt es sich bei dem Messergebnis, das von Interesse ist, beim Drehmomentschlüssel um den „Peak Torque“, der als Gleitkommazahl codiert, während es sich beim Messschieber bzw. Messuhr um „Measurement Distance“ als Ganzzahl codiert handelt. Im Fall der Messuhr wird das binäre Messergebnis erst als Ganzzahl intrepretiert, da es sich um die Distanz in Mikrometern handelt und anschließend auf Millimeter umgerechnet. Es wird als Nächstes überprüft, ob es sich bei der derzeitige Einheit des Geräts um inch handelt, da dann der Wert erneut durch 1000 dividiert werden muss, um den korrekten Umrechnungsfakor zu erhalten und von Mikroinch auf Inch zu gelangen. Es wird letztendlich eine Gleitkommazahl erhalten, wodurch sie der Nachrichten Struktur gespeichert werden kann und diese nicht angepasst werden muss.\\
Die Einheitenkodierung der Messuhr ist komplementär zur Kodierung der Drehmomentschlüssel und daher kann die if-Cascade, die die Zuordnung vornimmt um die Einheiten der Messuhr erweitert werden. Jedoch befindet sich die Information welche Einheit verwendet wird, wie das Messergebnis, ebenfalls an einer anderen Stelle innerhalb des Datenblocks.

\subsection{Gruppenfunktion}
Während im Modus 2 (CDC) die Messergebnisse von mehreren, durch Betätigung des Fußschalters getriggerten Messungen, anhand der Channelnummer einem Werkzeug zugeordnet werden können, ist dies in den HID-Modi nicht der Fall. Daher wird das devices.csv Konfigurationsfile um eine Spalte mit einer Gruppennummer erweitert. Werden die Geräte durch Betätigung des Fußschalters als Gruppe getriggert, werden die Messerergebnisse so sortiert, dass sie gemäß dieser Gruppenreihenfolge ausgegeben werden, wodurch sie wieder einer Messuhr zuordbar sind. Dazu bedarf es einem Counter, der durch Betätigung des Fußschalters, von 0 auf 1 gesetzt wird, wodurch der Start der Gruppenfunktion später erkennbar ist. Bei der Abarbeitung der erhaltenen Nachrichten, wird dann über die Zuordnung zum Device, die Nachricht ausgewählt und weitergegeben, die zum Counter korrespondiert. Die Gruppenids, die keinem der konfigurierten Geräte zugeordnet werden können, sowie die Gruppenids die unverbundenen Geräte gehören, werden dabei übersprungen. Zusätzlich soll ein Feature der Messeruhren genutzt werden, um die Gruppennummer auch auf der Messuhr anzuzeigen. Dazu werden den Messuhren ihre Gruppennummer nach dem Verbindungsaufbau via dem HCT-Protokoll übermittelt. Durch spätere Test und Anregungen von Messtechniker ergab sich außerdem, dass wenn ein Gerät der Gruppe zwar konfiguriert, jedoch nicht verbunden ist, es besser ist die gesamte Gruppe nicht zu triggern. Das ergibt sich einerseits dadurch, dass durch ihre relativ kleine Batterie der Messuhren, sich bei Inaktivität schnell ausschalten und die Verbindung zu ihrem Central trennen. Anderseits jedoch soll möglichst jede Messung korrekt sein, da sie in der CAQ-Software verarbeitet und gespeichert wird. Das Ziel der Sicherstellung einer korrekten Messung überwiegt hier also dem Gedanken der Benutzerfreundlichkeit, dass eine Messung auch dann gemacht werden kann, wenn einer der Geräte unverbunden ist. Der Fußschalter blinkt dann lediglich zwei kurz rot auf, um diesen Fehler zu signalisieren.