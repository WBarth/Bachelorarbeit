\section{Einbindung Messuhren}
In Vorbereitung auf die Implementierung der fußschalterspezifischen Features wird die Dongle-App um die Unterstützung der Messuhren bzw. Messschieber erweitert, da sich die meiste Funktionalität des Fußschalters spezielle auf die Messuhren bezieht. Messuhren und Messschieber sind in diesem Kontext dabei gleich bedeutend, weil das HCT-Interface für beide Geräte identisch ist. Es wird sich nachfolgend jedoch stets auf Messuhren bezogen, da sie für die Anwendungsfälle des Fußschalters und Dongles bedeutender sind. Bisher ist die Dongle-App daraufhin ausgelegt, dass mit einem Drehmomentschlüssel kommuniziert wird und die erhaltenen Binärdaten werden entsprechen interpretiert. Für die Einbindung der Messuhren muss nun die Unterscheidung eingeführt werden mit welcher Art von Werkzeug kommuniziert wird und entsprechend die Datenverarbeitung angepasst werden.

\subsection{Unterscheidung Geräte}
Die Unterscheidung mit welcher Art von Gerät kommuniziert wird, kann auf zwei verschiedenen Weisen erfolgen. \\
Einerseits kann anhand des Namens das Gerät entweder der Klasse Drehmomentschlüssel oder Messuhr zugeordnet werden. Das ist jedoch unter Umständen fehleranfällig, da das Sortiment an HCT-Werkzeug stetig wächst und es nicht auszuschließen ist, dass in der Zukunft Geräte auf den Markt kommen, mit deren Namen es zu falschen Zuordnung kommt. Zudem gibt es alte Messschieber, die zwar unter einem korrekten Name advertisen aber nicht das HCT-Protokoll sprechen, wobei diese Geräte aufgrund der Inkompatibilität der Protokolle letztendlich nicht verbunden werden sollten. \\
Eine andere Methode ist, die ``Device Information'' abzufragen und anhand der Klassenidentifikationsnummer das Gerät einem Typ zuzuordnen. Das ist die präferierte Lösung, da neben der genaueren und sicheren Zuordnung, in der Device Information andere Informationen, wie die Protokoll Version mitgeliefert werden, die bei späteren Features unter Umständen benötigt werden. Zudem bleibt die Datenverarbeitung leichter erweiterbar um neue Geräte. Jedoch muss diese Abfrage in den asynchronen und mehrteiligen Ablauf des Verbindungsaufbaus eingefügt werden, wodurch ein höherer Entwicklungsaufwand entsteht. Auch benutzt das Werkzeug der Firma Garant eine andere Klassifikation der Geräte als das Werkzeug der Firma Holex, welche innerhalb der Dongle-App vereinheitlicht werden muss.\\
Da die Erweiterbarkeit und Flexibilität jedoch dem höheren Entwicklungsaufwands überwiegen, wurde sich zugunsten dieser Lösung entschieden. Dabei wird nach dem Verbindungszustand-Callback, der vom Central Device nach einer erfolgreichen Subscription der Anwendung auf die BLE-Charakteristik des Werkzeugs aufgerufen wird, eine Nachricht zu Abfrage der Device Information gesandt. Die erhaltenen Daten werden dann in der zum Gerät gehörenden Struktur gespeichert und anschließend die Abfrage der Messeinheit durchgeführt.

\subsection{Anpassung der Datenverarbeitung}
Auch die Datenverarbeitung, also die Interpretation der erhalten Daten über BLE muss angepasst werden, da das Messergebnis zwar innerhalb des gleichen Datenblocks wie bei dem Holex Drehmomentschlüssel liegt, mit dem sie sich den gleichen Daten-Callback teilt, jedoch innerhalb des Datenblocks an einer anderen Stelle. Diese Offsets sind als Konstanten im Headerfile der Dongle-App definiert und müssen um die entsprechenden Einträge für die Messuhren erweitert werden. Zudem handelt es sich bei dem Messergebnis, das von Interesse ist, beim Drehmomentschlüssel um den ``Peak Torque'', der als Gleitkommazahl codiert ist, während bei der Messuhr das Ergebnis die ``Measurement Distance'' als Ganzzahl codiert ist. Dabei handelt sich abhängig von der Messeinheit um Mikrometer oder Mikroinch. Es muss auf Millimeter beziehungsweise Inch umgerechnet werden, da die Ausgabe über CDC oder HID der Anzeige auf dem Gerätedisplay gleichen soll. Das binäre Messergebnis wird daher zunächst mit einem Umrechnungsfaktor von 1000 in Millimeter beziehungsweise Milliinch als Gleitkommazahl umgerechnet. Anschließend muss überprüft werden, ob es sich bei der derzeitige Einheit des Geräts um inch handelt, da dann der Wert erneut durch 1000 dividiert werden muss, um von Milliinch auf die gewünschte Einheit Inch zu gelangen. Es wird also letztendlich eine Gleitkommazahl erhalten, wodurch sie ohne Anpassungen in der Nachrichten Struktur der Dongle-App gespeichert werden kann.\\
Die Einheitenkodierung der Messuhr ist komplementär zur Kodierung der Drehmomentschlüssel und daher kann die if-Cascade, welche die Zuordnung vornimmt, um die Einheiten der Messuhr erweitert werden. Jedoch befindet sich die Information welche Einheit verwendet wird, wie das Messergebnis, ebenfalls an einer anderen Stelle innerhalb des Datenblocks und muss entsprechend angepasst werden.

\subsection{Gruppenfunktion}
Durch das Betätigen des Fußtasters des Fußschalters soll bei allen verbundenen Messuhren das derzeitige Messergebnis abgefragt werden. Während im Modus 2 (CDC) ein Messergebnisse, anhand der Kanalnummer einem Werkzeug zugeordnet werden kann, ist dies in den HID-Modi nicht der Fall. Es muss daher die korrekte Reihenfolge der Ausgabe der Messergebnisse sichergestellt werden. Es wurde sich entschieden, das devices.csv Konfigurationsfile um eine Spalte mit einer Gruppennummer zu erweitern, da somit der Anwender sowohl die Reihenfolge als auch welche Messuhren in der Gruppe sind, konfigurieren kann. Da durch das Abschicken der Nachrichten zur Abfrage der Messungen in der korrekten Reihenfolgen, das tatsächliche Erhalten in der gleichen Reihenfolge nicht sichergestellt ist, muss davon ausgegangen werden, dass die Nachrichten in einer zufälligen Reihenfolgen erhalten werden. Stattdessen muss bei der Ausgabe die Nachrichten umsortiert werden. Dazu bedarf es eines Zählers, der durch Betätigung des Fußschalters, von 0 auf 1 gesetzt wird, wodurch der Start der Gruppenfunktion später beim Erhalten der Messergebnisse erkennbar ist. Bei der Abarbeitung der erhaltenen Nachrichten, wird dann über die Zuordnung zum Device, die Nachricht ausgewählt und weitergegeben, die zum Counter korrespondiert. Die Gruppenids, die keinem der konfigurierten Geräte zugeordnet werden können, sowie die Gruppenids die zu unverbundenen Geräten gehören, werden dabei übersprungen. Zusätzlich soll ein Feature der Messeruhren genutzt werden, um die Gruppennummer auch auf der Messuhr anzuzeigen. Dazu werden den Messuhren ihre Gruppennummer nach dem Verbindungsaufbau via dem HCT-Protokoll übermittelt.\\
Durch spätere Anregungen von Messtechnikern ergab sich jedoch, dass wenn ein Gerät der Gruppe zwar konfiguriert, jedoch nicht verbunden ist, die präferierte Lösung ist die gesamte Gruppe nicht zu triggern. Das ergibt sich einerseits dadurch, dass die Messuhren sich aufgrund ihrer relativ kleine Batterien bei Inaktivität schnell ausschalten und die Verbindung zu ihrem Central trennen. Anderseits soll möglichst jede durchgeführte Messung korrekt sein, da sie durch die CAQ-Software verarbeitet und gespeichert wird. Das Ziel der Sicherstellung einer korrekten Messung überwiegt hier also dem Gedanken der Benutzerfreundlichkeit, dass eine Messung auch dann gemacht werden kann, wenn einer der Geräte unverbunden ist. Der Fußschalter blinkt dann zwei Mal kurz rot auf, um diesen Fehler zu signalisieren.\\
Des Weiteren könnte es passieren, dass eine Messung zwar angefragt, aber nicht erhalten wurde. Die Anwendung würde dann blockieren, da auf die Nachricht gewartet wird und müsste neugestartet werden. Es wurde sich entschieden, einen Timer zu starten, falls eine Nachricht nicht erhalten wurde und bei seinem Ablaufen statt dem Messergebnis eine Fehlermeldung auszugeben. Die restlichen Messergebnisse können dann korrekt ausgegeben werden.\\
Ebenfalls aus dem Feedback der Messtechnikern heraus, wurde ein sequentielles Triggern der Gruppe implementiert. Dazu muss der entsprechende Eintrag in der Konfigurationsdatei mit einer Eins auf aktiv gesetzt werden. Es wird bei den Geräten der Gruppe nacheinander, jeweils bei Betätigen des Fußtaster, die Abfrage des Messergebnis ausgegelöst.