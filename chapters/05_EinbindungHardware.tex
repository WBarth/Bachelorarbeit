\section{Einbindung Hardware}
Die Prototypen des Fußschalters der Firma Brecht wurden bereits vor Beginn dieser Arbeit erhalten und somit konnte direkt mit der Inbetriebnahme der neuen Hardware begonnen werden. Der Chipsatz des Fußschalters ist ebenfalls der PCA10056 von Nordic semiconductor, somit muss an der Software des \ac{USB}-Dongles keine Änderungen vorgenommen werden. In diesem Kapitel werden die hardwarebezogenen Funktionalitäten des Energiemanagements, des Fußtaster und der \ac{LED} vorgestellt. Im Anhang finden sich die Hardwarezeichnungen der Platine.

\subsection{Energie Management}
Wenn der Fußschalter nicht über \ac{USB} mit einer Stromquelle verbunden ist, bekommt er den benötigten Strom von dem fest eingebauten Akku. Um diesen nicht unnötig zu belasten, muss ein Energiemanagement geschaffen werden, dass die bestehenden Ressourcen optimiert. Die Erhöhung der Effizienz der Batterienutzung kann dabei auf Weisen erreicht werden. Einerseits kann bei Nutzung des Fußschalter die Energieeffizienz verbessert werden. So wurde bereits, wie in Kapitel 3.3 beschrieben, eine Optimierung der Abfrage der Messeinheit durchgeführt. Andererseits kann bei Inaktivität des Fußschalters dessen Funktionalität abgeschaltet werden, um ebenfalls die Laufzeit der Batterie zu verlängern. Dazu muss festgelegt werden was Innaktivität bedeutet und nach welcher Zeit der Innaktivität drastische Stromsparmaßnahmen, wie das vollständige Ausschalten des Fußtasters, erfolgen.\\

Um diese Anforderung zu erfüllen, sah eine erste Idee vor das im nrf\_Base Projekt vorhandene Energiemanagement zu benutzen. Dieses fährt aus dem Main-Loop heraus getriggert den Chip bei Inaktivität des Softdevice weitestgehend herunter, wodurch der Stromverbrauch stark sinkt. Ist der \ac{USB}-Port also nicht verbunden und es wurde keine Aktivität in der Fußschalter Anwendung, wie das Erhalten einer Nachricht oder dem Verbinden eines Geräts, registriert, wird ein Timer gestartet. Läuft dieser Timer ab, werden alle Central und Peripheren Verbindungen getrennt und falls notwendig Scanning und Advertising gestoppt. Wird während dem Laufen des Timers Aktivität registriert, wird er neugestartet, während er vollständig gestoppt wird, falls \ac{USB} wieder verbunden ist. Dieser Implementierung lag die Vermutung zugrunde, dass wenn Chip vollständig heruntergefahren wurde, er auch nicht durch Betätigung des Fußtasters wieder neugestartet werden kann.\\
Zunächst zeigte sich jedoch das Problem, dass bei der Hardware des Fußschalters, der Akku auf der Datenleitung des \ac{USB} liegt. Dadurch wird in der Anwendung nicht wie bei dem EvalBoard die Events für \ac{USB} connected und \ac{USB} disconnect erhalten. Daher musste am Fußschalter geringfügige Hardwareänderungen durchgeführt werden. Dabei wurde die Eingangsspannung bereits vorher abgegriffen und auf den PIN des Fußtasters gelegt. Der Fußtaster erhält einen unbelegten PIN. Mit einem ADC wird dann überprüft, ob eine Spannung auf diesem PIN anliegt.\\
Es zeigte sich bei der Einbindung des Fußtasters, dass die Anwendung durch Betätigung des Fußtasters selbst aus einem Shutdown heraus wieder aufgeweckt wird. Daher wird nach Ablaufen des Inaktivitätstimers die Anwendung vollständig heruntergefahren, was noch energieeffizienter ist.

\subsection{Fußtaster Funktionalität}
Nach Überarbeitung der Codebasis und Einbindung der Messuhren, stehen alle Funktionalitäten bereit um den eigentlichen Fußtaster einzubinden. Während bei der initialen Einbindung des Fußtasters, eine Betätigung das Abfragen der Messergebnisse bei allen verbundenen Messuhren triggerte, wuchs die Funktionalität stetig.\\
Zum Ende dieser Arbeit wird durch ein kurzes Betätigen oder einem ``einfachen Klick'' die bereits beschriebene Gruppenfunktion ausgeführt. Im Modus 0 (USB-\ac{HID}-singleKey) und 4 (\ac{BLE}-\ac{HID}-singleKey) wird draufhin ein Timer gestartet, welcher derzeit auf 500ms gesetzt ist. Wird während seines Laufen eine zweite Betätigung getätigt, wird ein ``Doppelklick'' registriert und das dafür in der Konfigurationsdatei hinterlegte Zeichen ausgegeben. Dadurch kann der User in diesem Modus schnell Dialogoptionen in der \ac{HCT}-Windows-App auswählen oder in einem Textfile die Seiten wechseln. In einem anderen Modus führt das Warten auf die zweite Betätigung, jedoch zu einer Verzögerung der Ausgabe um die Dauer des Timers, weshalb dort die Doppelklick Funktionalität zugunsten dem Ansprechverhalten des Tasters gestrichen wurde.\\
Wird der Taster hingegen für 3 Sekunden durchgängig gehalten, wird das Gerät heruntergefahren. Diese Funktionalität wurde ursprünglich eingeführt, da der Fußschalter auf einer Messe vorgestellt wurde, aber der automatische Reset des Geräts bei Änderung der Konfigurationsfiles noch nicht implementiert war. Sie wurde aufgrund von positiven Feedback beibehalten.

\subsection{Inbetriebnahme LED}
Auf dem Board des Fußschalters befindet sich eine \ac{LED} die durch einen Lichtkanal nach außen hin durch das Gehäuse sichtbar gemacht wird und die dazu benutzt werden soll den internen Zustand des Geräts darzustellen. Folgende Zustände sollen abgebildet werden: 

\begin{table}[H]
	\centering
	\begin{tabular}[H]{l|l}
		Zustand & \ac{LED}-Farbe \\
		\hline
		Gerät im Sleep Modus & Aus \\
		\hline
		Alle zu verbindenden Geräte verbunden & Blau \\
		\hline
		Min. ein Gerät verbunden, es wird nach den fehlenden Geräten gescannt & Blau blinkend \\
		\hline
		Kein Gerät verbunden, Scanning inaktiv & Grün \\
		\hline
		Kein Gerät verbunden, Scanning aktiv & Grün blinkend \\
		\hline
		\ac{MSC}-Schreibvorgang detektiert & Gelb \\
		\hline
		Min. ein Konfigurationsfile nicht gefunden & Rot \\
		\hline
		Fehler in den Konfigurationsfiles & Rot blinkend \\
	\end{tabular}
	\caption{LED-Zustände}
\end{table}

Zudem blinkt der Fußschalter, wie in Kapitel 4.3 beschrieben, zwei mal kurz rot auf, wenn bei Betätigung des Fußtasters nicht alle zu der Gruppe gehörenden Geräte verbunden sind. Außerdem blinkt die Drei-Farben-\ac{LED} des Fußschalters vor einem Neustart oder dem Ausschalten des Geräts zweimal kurz grün.