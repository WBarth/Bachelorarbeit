\begin{abstract}

\section*{Abstract}
Die Digitalisierung von Hand- und Messwerkzeug in der Industrie führt dazu, dass eine steigende Menge an Daten zusammengeführt und verarbeitet werden muss. Für die Anbindung des Werkzeugs an einen Computer existieren verschiedene Möglichkeiten, die alle teils gravierende Nachteile besitzen. Zudem erfordern die unterschiedliche Anwendungsfälle für die Weiterverarbeitung und Speicherung die Daten der Messungen in verschiedenen Formaten. Dabei wird die folgende Forschungsfrage gestellt: Wie sehen die Komponenten eines drahtlosen Fußschalters aus, der die gängigen Anwendungsfälle der Anbindung von Messwerkzeug an einen Computer abdeckt? Es wird eine Anwendung entwickelt, sowohl auf einem drahtlosen Fußschalter, als auch einem USB-Dongle, die Daten von Messwerkzeugen zusammenführt und in verschiedenen konfigurierbaren Formaten dem Anwender zur Verfügung stellt. Diese Anwendung stellt einen neuen softwarebasierten Ansatz da, wie industrielles Messwerkzeug an einen Computer angebunden werden kann und die Messdaten verarbeitet werden können. Diese Anwendung wird von der Hoffmann Group, für deren Werkzeug exemplarisch die Entwicklung durchgeführt wurde, als Fußschalter und USB-Dongle in ihr Produktportfolio aufgenommen, während die Forschung die Ergebnisse dazu verwenden kann um weitere Konzepte zur Anbindung von Werkzeug an Computer zu entwickeln.

\end{abstract}