\begin{abstract}

\section*{Abstract}
Die Digitalisierung von Hand- und Messwerkzeugen in der Industrie führt dazu, dass eine steigende Menge an Daten zusammengeführt und verarbeitet werden muss. Sie werden dazu in speziellen Formaten benötigt, weshalb entweder Software- oder Hardware-Lösungen für die Umwandlung von Nöten sind. Während die Software-Lösungen, aufgrund der hohen Sicherheitsrichtlinien in der industriellen Fertigung, zeitaufwändige und komplexe Einführungsprozesse nachsichziehen, gibt es noch keine Hardware-Lösungen, welche sowohl die Umwandlung in die benötigten Formaten, als auch die kabellose Anbindung des Hand- und Messwerkzeugen übernimmt. Dabei wird die folgende Forschungsfrage gestellt: Wie sollten die Software- und Hardware-Komponenten eines intelligenten Fußschalters gestaltet werden, damit dieser ohne die Installation weiterer Software Hand- und Messwerkzeug an ein Computersystem kabellose anbindet und deren Messergebisse in verschiedenen konfigurierbaren Formaten zur Verfügung stellt? Es wird auf einem intelligenten Fußschalter, als auch auf einem USB-Dongle, eine Anwendung entwickelt, welche die Daten von Messwerkzeugen zusammenführt und für die Weiterverarbeitung aufbereitet. Diese Anwendung stellt einen neuen softwarebasierten Ansatz da, wie industrielle Hand- und Messwerkzeug an einen Computer angebunden werden kann und die Messdaten verarbeitet werden können. Sie kann von der Industrie dazu verwendet werden um darauf aufbauend konkrete Produkte zu entwickeln. So wird die Hoffmann Group, für deren Werkzeug exemplarisch die Implementierung durchgeführt wurde, den Fußschalter und USB-Dongle in ihr Produktportfolio aufnehmen. Die Ergebnisse der Entwicklung werden evaluiert und werden von der Forschung dazu verwendet, um weitere Konzepte zur Anbindung von Werkzeugen an Computer zu entwickeln.

\end{abstract}